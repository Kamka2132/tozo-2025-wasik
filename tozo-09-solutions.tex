\documentclass{article}
\usepackage[utf8]{inputenc}
\usepackage{polski}
\usepackage[polish]{babel}
\usepackage{amsmath}
\usepackage{amssymb}

\title{Teoria obliczeń i złożoność obliczeniowa 2025}
\author{Kamila Wasik}
\date{October 2025}

\begin{document}

\maketitle

\section{Zadanie 1}
Język
\[
\mathrm{EQ_{CFG}}=\{\langle G_1,G_2\rangle sa CFG \mid L(G_1)=L(G_2)\}
\]
jest co-rozpoznawalny.


Aby wykazać, że $\mathrm{EQ_{CFG}}$ jest co-rozpoznawalny, musimy wykazać, że jego dopełnienie $\overline{\mathrm{EQ_{CFG}}}$ (problem Nierówności Języków) jest rozpoznawalny w sensie Turinga.

\[
\overline{\mathrm{EQ_{CFG}}}=\{\langle G_1,G_2\rangle \mid L(G_1)\neq L(G_2)\}.
\]

 Jeśli $L(G_1) \neq L(G_2)$, to musi istnieć co najmniej jedno słowo $w$, które należy do jednego języka, ale nie do drugiego.

Maszyna Turinga $M$ (rozpoznająca $\overline{\mathrm{EQ_{CFG}}}$):

\begin{enumerate}
    \item Generowanie słów - maszyna $M$ generuje w nieskończonej pętli wszystkie możliwe słowa $w \in \Sigma^*$ w kolejności rosnącej długości (np. $\epsilon, 0, 1, 00, 01, \dots$).
    
    \item Testowanie Przynależności - dla każdego wygenerowanego słowa $w$, $M$ sprawdza:
    \begin{itemize}
        \item Czy $w \in L(G_1)$?
        \item Czy $w \in L(G_2)$?
    \end{itemize}
    \quad Możemy to zrobić, ponieważ problem $A_{CFG}$ jest rozstrzygalny.
    
    \item Decyzja o Nierówności:
    \begin{itemize}
        \item Jeżeli wyniki obu testów są różne, to $w$ jest świadkiem nierówności. Maszyna $M$ AKCEPTUJE $\langle G_1,G_2\rangle$ i zatrzymuje się.
        \item W przeciwnym razie ($L(G_1)$ i $L(G_2)$ dają ten sam wynik dla $w$), maszyna $M$ przechodzi do następnego słowa w enumeracji.
    \end{itemize}
\end{enumerate}

\textbf{Wniosek:}
\begin{itemize}
    \item Jeśli $L(G_1) \neq L(G_2)$, maszyna $M$ w końcu natrafi na słowo $w$ rozróżniające języki, zaakceptuje i się zatrzyma.
    \item Jeśli $L(G_1) = L(G_2)$, maszyna $M$ nigdy nie znajdzie różnicy i będzie kontynuować pracę w nieskończoność.
\end{itemize}
Ponieważ $M$ akceptuje i zatrzymuje się dla wszystkich wejść w $\overline{\mathrm{EQ_{CFG}}}$, dowodzi to, że $\overline{\mathrm{EQ_{CFG}}}$ jest rozpoznawalny. Zatem $\mathrm{EQ_{CFG}}$ jest co-rozpoznawalny.

\subsection*{Rola Postaci Normalnej Chomsky'ego (CNF)}

**Zastosowanie CNF** \\
CNF jest przydatna, ponieważ gwarantuje, że Test Przynależności (krok 2) zawsze zakończy się w skończonym czasie.

**Ograniczenie CNF** \\
CNF nie jest w stanie rozwiązać głównego problemu: konieczności przeszukania nieskończonej liczby słów. To dlatego $EQ_{CFG}$ pozostaje problemem nierozstrzygalnym.


\section{Zadanie 2}
Niech $B = \{\langle M, w \rangle \mid M$ jest dwutaśmową MT\}, 

która zapisuje niepusty symbol na swojej drugiej taśmie, gdy działa na  w. Wykaż, że $A_{TM}$ redukuje się do $B$.

Załóżmy przez sprzeczność, że Maszyna Turinga $R$ rozstrzyga $B$. Następnie skonstruujemy Maszynę Turinga $S$, która używa $R$ do rozstrzygania $A_{TM}$.

Konstrukcja Maszyny $S$ (Decydent dla $A_{TM}$):

$$S = \text{"Na wejściu } \langle M, w \rangle \text{ (gdzie } M \text{ jest MT, a } w \text{ jest słowem wejściowym):''}$$

\begin{enumerate}
    \item Konstrukcja Maszyny $T$ - użyj $M$ do skonstruowania następującej dwutaśmowej Maszyny Turinga $T$.
    \begin{quote}
    $T = \text{"Na wejściu } x$:
    \begin{enumerate}
        \item Symuluj $M$ na $x$ używając pierwszej taśmy.
        \item Jeżeli symulacja wykaże, że $M$ akceptuje, zapisz niepusty symbol na drugiej taśmie.
    \end{enumerate}
    \end{quote}
    
    \item Uruchomienie $R$ - uruchom $R$ na $\langle T, w \rangle$, aby określić, czy $T$ na wejściu $w$ zapisuje niepusty symbol na swojej drugiej taśmie.
    
    \item Decyzja $S$:
    \begin{itemize}
        \item Jeśli $R$ akceptuje, to $M$ akceptuje $w$, więc akceptuj.
        \item W przeciwnym razie, odrzuć
    \end{itemize}
\end{enumerate}

\section{Zadanie 3}
Niech C = $\{\langle M \rangle \mid \text{M  jest dwutaśmową MT}\}$, 
która zapisuje niepusty symbol na swojej drugiej taśmie, gdy działa na pewnym wejściu. Wykaż, że $A_{TM}$ redukuje się do $C$.

Załóżmy przez sprzeczność, że Maszyna Turinga $R$ rozstrzyga $C$. Skonstruuj Maszynę Turinga $S$, która używa $R$ do rozstrzygania $A_{TM}$.

$$S = \text{"Na wejściu } \langle M, w \rangle \text{ (gdzie } M \text{ jest MT, a } w \text{ jest słowem wejściowym):''}$$

\begin{enumerate}
    \item Konstrukcja Maszyny $T_w$ - użyj $M$ i $w$ do skonstruowania następującej dwutaśmowej Maszyny Turinga $T_w$.
    \begin{quote}
    $T_w = \text{"Na dowolnym wejściu}$:
    \begin{enumerate}
        \item Symuluj $M$ na $w$ używając pierwszej taśmy.
        \item Jeżeli symulacja wykaże, że $M$ akceptuje, zapisz niepusty symbol na drugiej taśmie.
    \end{enumerate}
    \end{quote}
    
    \item Uruchomienie $R$ - uruchom $R$ na $\langle T_w \rangle$, aby określić, czy $T_w$ kiedykolwiek zapisuje niepusty symbol na swojej drugiej taśmie.
    
    \item Decyzja $S$:
    \begin{itemize}
        \item Jeśli $R$ akceptuje, to $M$ akceptuje $w$, więc akceptuj
        \item W przeciwnym razie, odrzuć.
    \end{itemize}
\end{enumerate}

\section{Zadanie 4}
Twierdzenie Rice'a mówi, że każda nietrywialna własność języka rozpoznawalnego przez Maszynę Turinga (MT) jest nierozstrzygalna.



Oba warunki są potrzebne do dowodu, który opiera się na redukcji z problemu $A_{TM}$ (problem akceptacji, który jest nierozstrzygalny).

\subsection*{1) WARUNEK 1 ($P$ nie jest pusta)}

\textbf{Warunek:} Musi istnieć przynajmniej jedna Maszyna Turinga ($M_{TAK}$), której język $L(M_{TAK})$ posiada własność $P$.

\textbf{Uzasadnienie:}
Gdyby ten warunek nie był spełniony, to własność $P$ byłaby \textbf{trywialnie fałszywa} ($P = \emptyset$). Żaden język nie miałby tej własności.
W dowodzie Rice'a budujemy maszynę $S$, która ma rozstrzygać $A_{TM}$. Jeśli $M$ akceptuje $w$, maszyna $S$ musi zredukować to do przypadku, w którym nowy automat $T$ posiada własność $P$.

* Jeśli $P = \emptyset$, nie ma maszyny, do której moglibyśmy zredukować przypadek akceptacji. Nie mamy "celu" dla pozytywnej odpowiedzi.\\
* Język problemu $L_P$ byłby $\emptyset$, co jest językiem rozstrzygalnym.

\subsection*{2) WARUNEK 2 ($P$ nie jest zbiorem wszystkich języków)}

\textbf{Warunek:} Musi istnieć przynajmniej jedna Maszyna Turinga ($M_{NIE}$), której język $L(M_{NIE})$ nie posiada własności $P$.

\textbf{Uzasadnienie:}
Gdyby ten warunek nie był spełniony, to własność $P$ byłaby trywialnie prawdziwa ($P = \mathrm{RE}$, zbiór wszystkich języków rozpoznawalnych). Każdy język miałby tę własność.

W dowodzie Rice'a, jeśli $M$ nie akceptuje $w$, maszyna $S$ musi zredukować to do przypadku, w którym nowy automat $T$ nie posiada własności $P$.
\begin{itemize}
    \item Jeśli $P = \mathrm{RE}$, każda maszyna $T$ zawsze posiada własność $P$. Nie jesteśmy w stanie zredukować przypadku braku akceptacji do przypadku braku własności.
    \item Język problemu $L_P$ byłby $\Sigma^*$ (cały alfabet), co jest językiem rozstrzygalnym
\end{itemize}


\textbf{Podsumowanie:} Jeśli własność jest trywialna (obejmuje $0\%$ lub $100\%$ języków), odpowiedni problem jest łatwy (rozstrzygalny). Rice stwierdza, że cała trudność bierze się właśnie z konieczności podziału i rozróżnienia (nietrywialności) w tym nieskończonym zbiorze.

\section{Zadanie 5}
\subsection*{(a) $INF_{TM} = \{\langle M \rangle \mid L(M) \text{ jest językiem nieskończonym}\}$}

\begin{itemize}
    \item \textbf{Własność jest spełniona:} Istnieje MT $M_1$, która akceptuje język nieskończony. Np. $M_1$ akceptuje $\Sigma^*$ (wszystkie słowa). Język $\Sigma^*$ jest nieskończony.
    \item \textbf{Własność nie jest spełniona:} Istnieje MT $M_2$, która akceptuje język skończony. Np. $M_2$ akceptuje $\emptyset$ (język pusty) lub $\{0, 1\}$ (język zawierający tylko dwa słowa).
\end{itemize}
Własność jest nietrywialna. Z Twierdzenia Rice’a, $INF_{TM}$ jest nierozstrzygalny.

\subsection*{(b) $L_B = \{\langle M \rangle \mid 1011 \in L(M)\}$}

\begin{itemize}
    \item \textbf{Własność jest spełniona:} Istnieje MT $M_1$, która akceptuje wszystkie słowa, w tym $1011$. Np. $M_1$ akceptuje $\Sigma^*$.
    \item \textbf{Własność nie jest spełniona:} Istnieje MT $M_2$, która akceptuje język pusty $L(M_2)=\emptyset$, który nie zawiera słowa $1011$.
\end{itemize}
Własność jest nietrywialna. Z Twierdzenia Rice’a, $L_B$ jest nierozstrzygalny.

\subsection*{(c) $ALL_{TM} = \{\langle M \rangle \mid L(M) = \Sigma^*\}$}

\begin{itemize}
    \item \textbf{Własność jest spełniona:} Istnieje MT $M_1$, która akceptuje $\Sigma^*$. Np. $M_1$ natychmiast akceptuje każde wejście.
    \item \textbf{Własność nie jest spełniona:} Istnieje MT $M_2$, która akceptuje język pusty $L(M_2)=\emptyset$. Ponieważ $\emptyset \neq \Sigma^*$, $L(M_2)$ nie posiada tej własności.
\end{itemize}
Własność jest nietrywialna. Z Twierdzenia Rice’a, $ALL_{TM}$ jest nierozstrzygalny.
\\
\textbf{Wniosek:} Ponieważ dla wszystkich podpunktów spełnione są oba warunki nietrywialności, Twierdzenie Rice'a ma zastosowanie, a wszystkie te języki są nierozstrzygalne.

\section{Zadanie 6}
Niech $A_{\epsilon TM} = \{\langle M \rangle \mid M \text{ jest maszyną Turinga, która akceptuje słowo } \epsilon\}$.

Wykazujemy, że język $A_{\epsilon TM}$ jest nierozstrzygalny, przeprowadzając redukcję z problemu $A_{TM} = \{\langle M, w \rangle \mid M \text{ akceptuje } w\}$. Problem $A_{TM}$ jest znany jako nierozstrzygalny.

\subsection*{Konstrukcja redukcji $A_{TM} \le_m A_{\epsilon TM}$}

Skonstruujemy obliczalną funkcję $f$, która przekształca każde wejście $\langle M, w \rangle$ dla $A_{TM}$ w wejście $\langle T \rangle$ dla $A_{\epsilon TM}$ tak, że:
$$
\langle M, w \rangle \in A_{TM} \iff \langle T \rangle \in A_{\epsilon TM}
$$

Funkcja $f$ działa tak: na wejściu $\langle M, w \rangle$, $f$ konstruuje nową Maszynę Turinga $T$.

Konstrukcja Maszyny $T$:

Maszyna $T$ działa na dowolnym wejściu $x$ w następujący sposób:
\begin{enumerate}
    \item Sprawdzenie wejścia- jeżeli wejście $x$ nie jest słowem pustym ($\epsilon$), maszyna $T$ odrzuca.
    \item Dla $\epsilon$- jeżeli wejście $x$ jest słowem pustym ($\epsilon$), maszyna $T$ ignoruje wejście i symuluje maszynę $M$ na słowie $w$.
    \item Akceptacja - jeżeli symulacja $M$ na $w$ akceptuje, maszyna $T$ akceptuje. W przeciwnym razie $T$ zapętla się (lub odrzuca, jeśli $M$ odrzuca $w$).
\end{enumerate}
Formalnie $T$ akceptuje $x$ wtedy i tylko wtedy, gdy $x=\epsilon$ i $M$ akceptuje $w$.

\subsection*{Uzasadnienie poprawności redukcji}

Musimy sprawdzić równoważność: $\langle M, w \rangle \in A_{TM} \iff \langle T \rangle \in A_{\epsilon TM}$.

\begin{enumerate}
    \item przypadek $\implies$ - załóżmy, że $\langle M, w \rangle \in A_{TM}$ (tzn. $M$ akceptuje $w$).
    \begin{itemize}
        \item Ponieważ $M$ akceptuje $w$, zgodnie z krokiem 2 i 3 konstrukcji $T$, maszyna $T$ po wczytaniu $\epsilon$ uruchomi symulację, która się zakończy akceptacją.
        \item Zatem $T$ akceptuje $\epsilon$, co oznacza $\langle T \rangle \in A_{\epsilon TM}$.
    \end{itemize}
    
    \item przypadek $\impliedby$ - załóżmy, że $\langle T \rangle \in A_{\epsilon TM}$ (tzn. $T$ akceptuje $\epsilon$).
    \begin{itemize}
        \item Aby $T$ zaakceptowała $\epsilon$, musiała przejść przez krok 2 (symulacja $M$ na $w$) i ta symulacja musiała się zakończyć akceptacją (krok 3).
        \item Skoro $M$ akceptuje $w$, to $\langle M, w \rangle \in A_{TM}$.
    \end{itemize}
\end{enumerate}

\subsection*{Wniosek}



Skoro skonstruowaliśmy redukcję z nierozstrzygalnego języka $A_{TM}$ do języka $A_{\epsilon TM}$, i redukcja ta jest funkcją obliczalną (maszyna $T$ jest konstruowana w skończonym czasie), to $A_{\epsilon TM}$ jest również nierozstrzygalny.

Gdyby $A_{\epsilon TM}$ był rozstrzygalny przez maszynę $R$, moglibyśmy użyć $R$ do rozstrzygnięcia $A_{TM}$:
$$
S(\langle M, w \rangle) = R(f(\langle M, w \rangle))
$$
co prowadzi do sprzeczności.


\section{Zadanie 7}
Definicja problemu\\
Rozważmy język $W_{\mathrm{BLANK}}$ (Write Blank - Zapis Pustego Symbolu) złożony z kodowań maszyn Turinga, które w trakcie obliczeń dla dowolnego słowa wejściowego zamieniają niepusty symbol na pusty.

Niech $\Sigma$ będzie alfabetem wejściowym, a $\Gamma$ alfabetem taśmowym, przy czym $\sqcup \in \Gamma$ jest symbolem pustym (blank).

\[
W_{\mathrm{BLANK}} = \{\langle M \rangle \mid M \text{ jest jednotaśmową MT, która działając na pewnym wejściu }\]
\[
w \in \Sigma* \text{ i na pewnym kroku, zapisuje symbol } \sqcup \text{ w miejscu, gdzie wcześniej znajdował} \]
\[ 
\text{się symbol  x} \in \Gamma \setminus \{\sqcup\}\}
\]

\subsection*{Dowód nierozstrzygalności $W_{\mathrm{BLANK}}$}

Wykazujemy, że język $W_{\mathrm{BLANK}}$ jest nierozstrzygalny, przeprowadzając redukcję z problemu $A_{TM} = \{\langle M, w \rangle \mid M \text{ akceptuje } w\}$, który jest znany jako nierozstrzygalny.

\subsection*{Konstrukcja redukcji $A_{TM} \le_m W_{\mathrm{BLANK}}$}

Skonstruujemy funkcję obliczalną $f$, która przekształca wejście $\langle M, w \rangle$ dla $A_{TM}$ w wejście $\langle T \rangle$ dla $W_{\mathrm{BLANK}}$ tak, że:
$$
\langle M, w \rangle \in A_{TM} \iff \langle T \rangle \in W_{\mathrm{BLANK}}
$$

Funkcja $f$ konstruuje nową Maszynę Turinga $T$, która działa w następujący sposób:

Konstrukcja Maszyny $T$:

Maszyna $T$ działa na dowolnym wejściu $x \in \Sigma^*$ w następujący sposób:
\begin{enumerate}
    \item Przygotowanie taśmy - $T$ na początku przechodzi przez całe wejście $x$ i zamienia każdy symbol $x_i$ na specjalny symbol $x'_i$, który nie jest pusty ($\sqcup$) i który jest oznaczony jako niemożliwy do wymazania (np. $x_i$ zamieniamy na $\tilde{x}_i \in \Gamma \setminus \{\sqcup\}$).
    \item Symulacja - maszyna $T$ ignoruje wejście $x$ (które jest już na taśmie w postaci $\tilde{x}$), a następnie symuluje maszynę $M$ na słowie $w$. Symulacja odbywa się na wolnym obszarze taśmy (za słowem $\tilde{x}$).
    \item blokada zapisu pustego symbolu (klucz) -  maszyna $T$ jest tak zaprojektowana, aby w trakcie symulacji $M$ na $w$, $T$ nigdy nie zapisywała symbolu $\sqcup$ w miejscu, gdzie wcześniej znajdował się symbol niepusty. Jeśli symulacja wymagałaby wymazania symbolu, $T$ zapisuje w to miejsce nowy symbol specjalny, np. $\natural \in \Gamma \setminus \{\sqcup\}$.
    \item decyzja akceptacji - jeżeli symulacja $M$ na $w$ akceptuje, maszyna $T$ przesuwa się na początek taśmy i wymazuje pierwszy symbol wejścia (tj. zapisuje $\sqcup$ w miejscu, gdzie znajdował się $\tilde{x}_1 \in \Gamma \setminus \{\sqcup\}$). Następnie $T$ akceptuje.
    \item odrzucenie/zapętlenie - jeżeli symulacja $M$ na $w$ odrzuca lub się zapętla, maszyna $T$ również odrzuca lub zapętla się, nigdy nie dokonując operacji wymazania.
\end{enumerate}

\subsection*{Uzasadnienie poprawności redukcji}

Musimy sprawdzić równoważność: $\langle M, w \rangle \in A_{TM} \iff \langle T \rangle \in W_{\mathrm{BLANK}}$.

\begin{enumerate}
    \item Przypadek $\implies$ - załóżmy, że $M$ akceptuje $w$.
    \begin{itemize}
        \item Maszyna $T$ dla dowolnego wejścia $x$ przeprowadzi symulację $M$ na $w$.
        \item Symulacja zakończy się akceptacją, co uruchomi krok 4, w którym $T$ zapisze $\sqcup$ w miejscu $\tilde{x}_1$, gdzie wcześniej był symbol niepusty.
        \item Zatem $\langle T \rangle \in W_{\mathrm{BLANK}}$.
    \end{itemize}
    
    \item Przypadek $\impliedby$ - załóżmy, że $\langle T \rangle \in W_{\mathrm{BLANK}}$ (tzn. $T$ zapisuje $\sqcup$ w miejscu symbolu niepustego dla pewnego wejścia $x$).
    \begin{itemize}
        \item Zgodnie z konstrukcją $T$ (kroki 1, 3, 5), jedynym momentem, w którym $T$ zapisuje $\sqcup$ w miejscu symbolu niepustego, jest krok 4.
        \item Krok 4 jest wykonywany wyłącznie, gdy symulacja $M$ na $w$ zakończy się akceptacją.
        \item Zatem $M$ akceptuje $w$, co oznacza $\langle M, w \rangle \in A_{TM}$.
    \end{itemize}
\end{enumerate}



\subsection*{Wniosek}

Ponieważ skonstruowaliśmy redukcję z nierozstrzygalnego języka $A_{TM}$ do $W_{\mathrm{BLANK}}$, dowodzi to, że $W_{\mathrm{BLANK}}$ jest nierozstrzygalny. Gdyby $W_{\mathrm{BLANK}}$ był rozstrzygalny, to $A_{TM}$ również musiałby być rozstrzygalny, co prowadzi do sprzeczności.

\section{Zadanie 8}

Stan $q$ maszyny Turinga $M$ jest bezużyteczny, jeśli dla żadnego słowa wejściowego $w$ maszyna $M$ nie przechodzi do stanu $q$ w trakcie swoich obliczeń.
\\
Rozważmy język $USELESS_{TM}$ złożony z kodowań maszyn Turinga, które posiadają co najmniej jeden stan bezużyteczny.
\[
USELESS_{TM} = \{\langle M \rangle \mid M \text{ jest MT i posiada co najmniej jeden stan bezużyteczny}\}.
\]

\subsection*{Dowód nierozstrzygalności $USELESS_{TM}$}

Wykazujemy, że język $USELESS_{TM}$ jest nierozstrzygalny, przeprowadzając redukcję z problemu $E_{TM} = \{\langle M \rangle \mid L(M) = \emptyset\}$, który jest znany jako nierozstrzygalny.

\subsection*{Konstrukcja redukcji $E_{TM} \le_m USELESS_{TM}$}

Skonstruujemy funkcję obliczalną $f$, która przekształca wejście $\langle M \rangle$ dla $E_{TM}$ w wejście $\langle T \rangle$ dla $USELESS_{TM}$ tak, że:
$$
\langle M \rangle \in E_{TM} \iff \langle T \rangle \in USELESS_{TM}
$$

Funkcja $f$ konstruuje nową Maszynę Turinga $T$ na podstawie maszyny $M$.

\subsection*{Konstrukcja nowej maszyny $T$}

Maszyna $T$ jest tworzona przez dodanie do maszyny $M$ jednego nowego, specjalnego stanu $q_{\mathrm{extra}}$, który jest stanem nieakceptującym i nierozpoczęcia. Maszyna $T$ ma ten sam język wejściowy i alfabet taśmowy co $M$.
Zasady przejścia maszyny $T$:
\begin{enumerate}
    \item $T$ zachowuje wszystkie reguły przejścia maszyny $M$.
    \item Dodajemy nowy, nieużywany stan $q_{\mathrm{extra}}$.
    \item Nie dodajemy żadnych reguł przejścia, które prowadziłyby do stanu $q_{\mathrm{extra}}$, ani nie wychodzą z tego stanu (z wyjątkiem ewentualnej reguły tożsamości $q_{\mathrm{extra}}, x \to q_{\mathrm{extra}}, x, S$, jeśli stan ma być w ogóle użyteczny, ale tutaj celowo robimy go niedostępnym).
\end{enumerate}

Maszyna $T$ działa na wejściu $x$:
\begin{enumerate}
    \item $T$ zaczyna w stanie początkowym $q_0$ i symuluje $M$ na wejściu $x$.
    \item Maszyna $T$ ma zbiór stanów $Q_T = Q_M \cup \{q_{\mathrm{extra}}\}$.
\end{enumerate}
Z definicji maszyny $T$, jedynym sposobem, aby $q_{\mathrm{extra}}$ został odwiedzony, byłoby istnienie reguły przejścia $\delta_M(q, x) = (q_{\mathrm{extra}}, y, d)$ lub $\delta_T(q, x) = (q_{\mathrm{extra}}, y, d)$. Ponieważ celowo nie dodajemy żadnych takich reguł, dostęp do $q_{\mathrm{extra}}$ jest możliwy tylko wtedy, gdy stan początkowy $q_0$ jest równy $q_{\mathrm{extra}}$, co zakładamy, że nie ma miejsca.

Aby uzyskać poprawną redukcję, musimy zapewnić, że $M$ ma przynajmniej jeden stan dostępny. Zakładamy, że $M$ jest poprawną MT.

Rozważmy ulepszoną konstrukcję $T$:

Konstrukcja $T$ (ulepszona):
$T$ ma zbiór stanów $Q_T = Q_M \cup \{q_{\mathrm{extra}}\}$.
\begin{enumerate}
    \item $T$ zachowuje wszystkie reguły przejścia z $M$.
    \item $T$ jest maszyną $M$, do której dodaliśmy nowy, niedostępny stan $q_{\mathrm{extra}}$.
\end{enumerate}

\subsection*{Uzasadnienie poprawności redukcji}

Musimy sprawdzić równoważność: $\langle M \rangle \in E_{TM} \iff \langle T \rangle \in USELESS_{TM}$.

\begin{enumerate}
    \item Przypadek $\implies$ - załóżmy, że $\langle M \rangle \in E_{TM}$ (tzn. $L(M) = \emptyset$).
    \begin{itemize}
        \item Ponieważ $L(M) = \emptyset$, maszyna $M$ nigdy nie akceptuje żadnego słowa wejściowego.
        \item Skonstruujmy $T$ tak, że $q_{\mathrm{extra}}$ jest stanem, do którego można przejść tylko po akceptacji przez $M$.
        \item W tym celu zmieniamy maszynę $T$ tak, aby $q_{\mathrm{accept}}$ maszyny $M$ przekierowywał do $q_{\mathrm{extra}}$.

        Korekta konstrukcji $T$ (klucz redukcji):
        $T$ zachowuje wszystkie reguły przejścia z $M$, ale usuwa stan akceptujący $q_{\mathrm{accept}} \in Q_M$. Zamiast niego, każda reguła $\delta_M(q, x) = (q_{\mathrm{accept}}, y, d)$ jest zastąpiona przez $\delta_T(q, x) = (q_{\mathrm{extra}}, y, d)$. Stan $q_{\mathrm{extra}}$ staje się nowym stanem akceptującym $T$. Dodatkowo $T$ ma drugi, nowy stan $q_{\mathrm{useless}}$, który jest niedostępny z żadnego innego stanu.

        \item W maszynie $T$:
            * Stany z $M$ są użyteczne $\iff M$ na jakimś wejściu dotrze do tego stanu.
            * Stan $q_{\mathrm{useless}}$ jest bezużyteczny $\iff$ Nie ma reguły prowadzącej do $q_{\mathrm{useless}}$.

        \item Skupmy się na prostszej konstrukcji: $T$ ma dwa nowe stany $q_{\mathrm{accept}}$ (stan akceptujący $T$) i $q_{\mathrm{useless}}$. Nie ma reguł prowadzących do $q_{\mathrm{useless}}$.

        \item Jeśli $L(M) = \emptyset$, to $M$ nigdy nie wejdzie do swojego stanu akceptującego. W nowej konstrukcji $T$ robimy redukcję do $E_{TM}$ tak, aby $q_{\mathrm{useless}}$ był dostępny tylko, gdy $L(M) \neq \emptyset$. To jest trudne, bo redukujemy z $E_{TM}$.

        Wracając do Redukcji $E_{TM} \le_m USELESS_{TM}$ (właściwa wskazówka):

        Niech $T$ ma stany $Q_T = Q_M \cup \{q_{\mathrm{extra}}\}$, gdzie $q_{\mathrm{extra}}$ jest nowym, niedostępnym stanem.
        Jeżeli $L(M) = \emptyset$, oznacza to, że $M$ (a zatem i $T$) nigdy nie wejdzie do stanu $q_{\mathrm{accept}}$. Wszystkie stany z $M$ są nadal użyteczne, o ile $M$ się na nich w ogóle zaczyna i wykonuje jakieś obliczenia.
        
        Konstrukcja $T$ (poprawne powiązanie):
        Dodajemy nowy stan $q_{\mathrm{useless}}$ oraz $q_{\mathrm{trigger}}$.
        1. $T$ symuluje $M$ na $x$.
        2. Jeżeli $M$ akceptuje, $T$ przechodzi do $q_{\mathrm{trigger}}$.
        3. Ze stanu $q_{\mathrm{trigger}}$, $T$ przechodzi do wszystkich stanów $q \in Q_M$ oraz \textbf{do $q_{\mathrm{useless}}$}.
        4. Jeżeli $M$ nie akceptuje, $T$ nigdzie nie przechodzi.

        *uprawomocnienie (najprostsza forma):
        $T$ ma stan $q_{\mathrm{useless}}$, który jest osiągalny \textbf{tylko} ze stanu akceptującego $M$.
        
        $T$ ma stany $Q_T = Q_M \cup \{q_{\mathrm{extra}}, q_{\mathrm{useless}}\}$.
        \begin{enumerate}
            \item Reguły $M$ są zachowane.
            \item Nowy stan $q_{\mathrm{extra}}$ jest stanem, do którego przechodzi $T$ po akceptacji $M$: $\delta_T(q, x) = (q_{\mathrm{extra}}, y, d)$, jeśli $\delta_M(q, x) = (q_{\mathrm{accept}}, y, d)$.
            \item Z $q_{\mathrm{extra}}$ jest reguła prowadząca TYLKO do $q_{\mathrm{useless}}$.
            \item $q_{\mathrm{useless}}$ jest stanem akceptującym $T$.
        \end{enumerate}

        \item W tym przypadku $L(M) = \emptyset$. $M$ nigdy nie akceptuje. $T$ nigdy nie przechodzi do $q_{\mathrm{extra}}$, a co za tym idzie,$q_{\mathrm{useless}}$ jest stanem bezużytecznym. Zatem $\langle T \rangle \in USELESS_{TM}$.
    \end{itemize}
    
    \item Przypadek $\impliedby$ - załóżmy, że $\langle T \rangle \in USELESS_{TM}$ (tzn. $T$ ma stan bezużyteczny $q_{\mathrm{bezużyteczny}}$).
    \begin{itemize}
        \item Jedynym stanem, który może być bezużyteczny, jest $q_{\mathrm{useless}}$, ponieważ wszystkie stany z $Q_M$ są stanami startowymi i są używane w symulacji, jeśli $M$ wykonuje jakikolwiek ruch.
        \item Jeżeli $q_{\mathrm{useless}}$ jest bezużyteczny, to nigdy nie został osiągnięty.
        \item Osiągnięcie $q_{\mathrm{useless}}$ wymaga przejścia przez $q_{\mathrm{extra}}$, co z kolei wymagało, aby $M$ akceptowała jakieś słowo $w$.
        \item Skoro $q_{\mathrm{useless}}$ nie został osiągnięty, to $M$ nigdy nie akceptuje żadnego słowa
        \item Zatem $L(M) = \emptyset$, co oznacza $\langle M \rangle \in E_{TM}$.
    \end{itemize}
\end{enumerate}


\subsection*{Wniosek}

Skoro skonstruowaliśmy redukcję z nierozstrzygalnego języka $E_{TM}$ do $USELESS_{TM}$, to $USELESS_{TM}$ jest również nierozstrzygalny.












\end{document}
