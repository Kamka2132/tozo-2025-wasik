\documentclass{article}
\usepackage[utf8]{inputenc}
\usepackage{polski}
\usepackage{amsmath}
\usepackage{amssymb}

\title{Teoria obliczeń i złożoność obliczeniowa 2025}
\author{Kamila Wasik}
\date{October 2025}

\begin{document}

\maketitle

\section{Wyrażenia regularne}
ZADANIE 1
\begin{enumerate}
    \item a*b*\\
    należą: ab, a, b\\
    nie należą: ba, bba
    \item a(ba)*b\\
    należą: ab, abab\\
    nie należą: b, baba
    \item a* $\cup$ b*\\
    należą: a, b \\
    nie należą: bab, baba
    \item (aaa)*\\
    należą: aaa, aaaaaa\\
    nie należą:b, aa, bab
    \item $\Sigma$*a$\Sigma$*b$\Sigma$*a$\Sigma$*\\
    należą: aba, aabaa\\
    nie należą:a, b, bab
    \item aba $\cup$ bab\\
    należą: aba, bab\\
    nie należą: a, b
    \item ($\varepsilon$ $\cup$ a) b\\
    należą: ab, b\\
    nie należą: a, aa, bab
    \item (a $\cup$ ba $\cup$ bb) $\Sigma$*\\
    należą: a, ba, bb, aba\\
    nie należą: a, $\varepsilon$, b
\end{enumerate}

\newpage
ZADANIE 2
\begin{enumerate}
    \item \{w $\in$ $\Sigma$* $|$ w zaczyna się symbolem 1 i kończy się symbolem 0\}\\
    odp: 1$\Sigma$*0
    \item \{w $\in$ $\Sigma$* $|$ w zawiera przynajmniej trzy symbole 1\}\\
    odp: $\Sigma$*1$\Sigma$*1$\Sigma$*1$\Sigma$*
    \item \{w $\in$ $\Sigma$* $|$ w zawiera podsłowo 0101\}\\
    odp: $\Sigma$*0101$\Sigma$*
    \item \{w $\in$ $\Sigma$* $|$ w ma długość co najmniej 3 i jego trzecim symbolem jest 0\}\\
    odp: $\Sigma$$\Sigma$0$\Sigma$*
    \item \{w $\in$ $\Sigma$* $|$ w zaczyna się symbolem 0 i ma nieparzystą długość lub zaczyna się symbolem 1 i ma parzystą długość\}\\
    odp: 0($\Sigma$$\Sigma$)* $\cup$ 1$\Sigma$($\Sigma$$\Sigma$)*
    \item \{w $\in$ $\Sigma$* $|$ w nie zawiera podsłowa 110\}\\
    odp: (0$\cup$(10)*)*1*
    \item \{w $\in$ $\Sigma$* $|$ w ma długość co najwyżej 5\}\\
    odp: ($\Sigma$$\cup$$\varepsilon$)$^5$
    \item \{w $\in$ $\Sigma$* $|$ w jest dowolnym słowem różnym od 11 oraz od 111\}\\
    odp: $\varepsilon$ $\cup$ $\Sigma$ $\cup$ 0$\Sigma$ $\cup$ 10 $\cup$ 0$\Sigma$$\Sigma$ $\cup$ 10$\Sigma$ $\cup$ 110 $\cup$ $\Sigma$$^3$$\Sigma$$^+$
    \item \{w $\in$ $\Sigma$* $|$ w na każdej nieparzystej pozycji w występuje symbol 1\}\\
    odp: (1$\Sigma$)*(1$\cup$$\varepsilon$)
    \item \{w $\in$ $\Sigma$* $|$ w zawiera co najmniej dwa symbole 0 oraz co najwyżej jeden symbol 1\}\\
    odp: (0 $\cup$ 1)*0 (0$\cup$1)*0 (0$\cup$1)* $\cap$ (0*$\cup$0*10*)
    \item \{$\varepsilon$, 0\}\\
    odp: ($\varepsilon$ $\cup$ 0)
    \item \{w $\in$ $\Sigma$* $|$ w zawiera parzystą liczbę symboli 0 lub zawiera dokładnie dwa symbole 1\}\\
    odp: 1*(01*0)*1* $\cup$ 0*10*10*
    \item $\varnothing$\\
    odp: Język pusty nie zawiera żadnego słowa
    \item \{w $\in$ $\Sigma$* $|$ w $\neq$ $\varepsilon$ (wszystkie słowa niepuste)\}\\
    odp: (0$\cup$1)(0$\cup$1)*
\end{enumerate}

\section{Gramatyki bezkontekstowe}
ZADANIE 1\\
a) symbole nieterminalne (zmienne): R, S, T, X\\
b) symbole terminalne: a, b\\
c) symbol początkowy: R\\
d) 3 słowa z języka L(G): aab, bba, ab\\
e) 3 słowa, które nie należą do tego języka: $\varepsilon$, aaa, b \\
f) Czy prawdą jest, że:\\
\begin{enumerate}
    \item T $\Rightarrow$ aba FAŁSZ
    \item T $\Rightarrow$* aba PRAWDA
    \item T $\Rightarrow$ T FAŁSZ
    \item T $\Rightarrow$* T PRAWDA
    \item XXX $\Rightarrow$* aba PRAWDA
    \item X $\Rightarrow$* aba FAŁSZ
    \item T $\Rightarrow$* XX PRAWDA
    \item T $\Rightarrow$* XXX PRAWDA
    \item S $\Rightarrow$* $\varepsilon$ FAŁSZ
\end{enumerate}
g) Gramatyka G generuje takie słowa, które można zapisać w postaci u s rev(u), gdzie:\\
u - dowolne słowo nad \{a,b\}\\
s - środkowy blok pochodzący z reguły S. Ma postać aTb lub bTa, przy czym T jest palindromem nad \{a,b\}\\
rev(u) - lustrzane odbicie u\\
Każdy ciąg składa się z symetrycznych części skrajnych oraz środkowego fragmentu, który zaczyna się i kończy różnymi literami, a między nimi jest palindrom (możliwy pusty).\\


ZADANIE 2\\
a) Zbiór słów wszystkich słów zawierających tyle samo a co b\\
S $\rightarrow$ aSb\\
S $\rightarrow$ bSa\\
S $\rightarrow$ SS\\
S $\rightarrow$ $\varepsilon$\\
b) Zbiór słów wszystkich słów zawierających co najmniej tyle samo a co b\\
S $\rightarrow$ aS\\
S $\rightarrow$ Sa\\
S $\rightarrow$ SS\\
S $\rightarrow$ E\\
E $\rightarrow$ aEb\\
E $\rightarrow$ bEa\\
E $\rightarrow$ EE\\
E $\rightarrow$ $\varepsilon$\\
c) Zbiór słów wszystkich słów zawierających więcej a niż b\\
S $\rightarrow$ aS\\
S $\rightarrow$ Sa\\
S $\rightarrow$ SS\\
S $\rightarrow$ E\\
E $\rightarrow$ aEb\\
E $\rightarrow$ bEa\\
E $\rightarrow$ EE\\
E $\rightarrow$ $\varepsilon$\\
T $\rightarrow$ a\\
T $\rightarrow$ aT\\
T $\rightarrow$ Ta\\
T $\rightarrow$ TT\\
T $\rightarrow$ Ea\\
T $\rightarrow$ aE\\

\section{Automaty ze stosem}

ZADANIE 1

b) (q$_1$, 0110, $\varepsilon$) $\rightarrow$ (q$_2$, 0110, $\$$)  $\rightarrow$ (q$_2$, 110, 0$\$$) $\rightarrow$ (q$_2$, 10, 10$\$$) $\rightarrow$ (q$_3$, 0, 10$\$$) $\rightarrow$ (q$_3$, $\varepsilon$, 0$\$$) $\rightarrow$ (q$_3$, $\varepsilon$, $\$$) $\rightarrow$ (q$_4$, $\varepsilon$, $\varepsilon$)\\

c) (q$_1$, 110011, $\varepsilon$) $\rightarrow$ (q$_2$, 110011, $\$$)  $\rightarrow$ (q$_2$, 10011, 1$\$$) $\rightarrow$ (q$_2$, 0011, 11$\$$) $\rightarrow$ (q$_2$, 011, 011$\$$) $\rightarrow$ (q$_2$, 11, 0011$\$$) $\rightarrow$ (q$_3$, 1, 0011$\$$) $\rightarrow$ (q$_3$, $\varepsilon$, 001$\$$) $\rightarrow$ (q$_3$, $\varepsilon$, 00$\$$) $\rightarrow$ (q$_3$, $\varepsilon$, 0$\$$) $\rightarrow$ (q$_3$, $\varepsilon$, $\$$) $\rightarrow$ (q$_4$, $\varepsilon$, $\varepsilon$)\\

d) (q$_1$, 101101, $\varepsilon$) $\rightarrow$ (q$_2$, 101101, $\$$)  $\rightarrow$ (q$_2$, 01101, 1$\$$)  $\rightarrow$ (q$_2$, 1101, 01$\$$)  $\rightarrow$ (q$_2$, 101, 101$\$$)  $\rightarrow$ (q$_2$, 01, 1101$\$$)  $\rightarrow$ (q$_3$, 1, 1101$\$$)  $\rightarrow$ (q$_3$, $\varepsilon$, 101$\$$)  $\rightarrow$ (q$_3$, $\varepsilon$, 01$\$$)  $\rightarrow$ (q$_3$, $\varepsilon$, 1$\$$) $\rightarrow$ (q$_3$, $\varepsilon$, $\$$) $\rightarrow$ (q$_4$, $\varepsilon$, $\varepsilon$)


\end{document}
