\documentclass{article}
\usepackage[utf8]{inputenc}
\usepackage{polski}
\usepackage{amsmath}
\usepackage{amssymb}

\title{Teoria obliczeń i złożoność obliczeniowa 2025}
\author{Kamila Wasik}
\date{October 2025}

\begin{document}

\maketitle

ZADANIE 1

a) Czy maszyna Turinga może kiedykolwiek zapisać na taśmie symbol pusty (blank)?\\
Tak. Maszyna Turinga może zapisać symbol pusty (blank), bo `blank` należy do alfabetu taśmowego i może być wynikiem funkcji przejścia.\\

b) Czy alfabet taśmowy może być identyczny z alfabetem wejściowym?\\
Nie. Zwykle $\Sigma$ $\subset$ $\Gamma$, bo $\Gamma$ musi zawierać też symbol pusty (blank), którego nie ma w $\Sigma$.\\

c) Czy głowica może być w tej samej pozycji w dwóch kolejnych krokach?\\ 
Zależy od modelu. Jeśli dopuszcza ruch „bez przesunięcia” (np. S), to tak. Jeśli tylko L i R, to nie.\\

d) Czy maszyna Turinga może mieć tylko jeden stan? \\
Tak, może mieć jeden stan, ale wtedy jest bardzo ograniczona - nie może rozróżniać etapów obliczeń.\\

ZADANIE 2
\begin{table}
    \centering
    \begin{tabular}{cccccc}
         & 0 & 1 & x & $\sqcup$ & $\#$\\
         $q_1$& ($q_2$, x, R) & ($q_3$, x, R) & ($q_R$, x, R) & ($q_R$, $\sqcup$, R) & ($q_8$, $\#$, R)\\
         $q_2$& ($q_2$, 0, R) & ($q_2$, 1, R) & ($q_R$, x, R) & ($q_R$, $\sqcup$, R) & ($q_4$, $\#$, R)\\
         $q_3$& ($q_3$, 0, R) & ($q_3$, 1, R) & ($q_R$, x, R) & ($q_R$, $\sqcup$, R) & ($q_5$, $\#$, R)\\
         $q_4$& ($q_6$, x, L) & ($q_R$, 1, R) & ($q_4$, x, R) & ($q_R$, $\sqcup$, R) & ($q_R$, $\#$, R) \\
         $q_5$& ($q_R$, 0, R) & ($q_6$, x, L) & ($q_5$, x, R) & ($q_R$, $\sqcup$, R) & ($q_R$, $\#$, R)\\
         $q_6$& ($q_6$, 0, L) & ($q_6$, 1, L) & ($q_6$, x, L) & ($q_R$, $\sqcup$, R) & ($q_7$, $\#$, L)\\
         $q_7$& ($q_7$, 0, L) & ($q_7$, 1, L) & ($q_1$, x, R) & ($q_R$, $\sqcup$, R) & $q_R$, $\#$, R)\\
         $q_8$& ($q_R$, 0, R) & ($q_R$, 1, R) & ($q_8$, x, R) & ($q_A$, $\sqcup$, R) & $q_R$, $\#$, R)\\
    \end{tabular}
    \label{tab:placeholder}
\end{table}

\newpage
ZADANIE 3


a) 11\\
$q_1$11  $\vdash$  x$q_3$1  $\vdash$  x1$q_3$$\sqcup$  $\vdash$  x1$\sqcup$$q_R$\\

b) 1$\#$1//
$q_1$1$\#$1  $\vdash$  x$q_3$$\#$1  $\vdash$  x$\#$$q_5$1  $\vdash$  x$q_6$$\#$x  $\vdash$  $q_7$x$\#$x  $\vdash$  x$q_1$$\#$x  $\vdash$  x$\#$$q_8$x  $\vdash$  x$\#$x$q_8$$\sqcup$  $\vdash$  x$\#$x$\sqcup$$q_A$\\

c) 1$\#$$\#$1\\
$q_1$1$\#$$\#$1  $\vdash$  x$q_3$$\#$$\#$1  $\vdash$  x$\#$$q_5$$\#$1 $\vdash$   x$\#$$\#$$q_R$\\

d) 10$\#$11\\
$q_1$10$\#$11 $\vdash$ x$q_3$0$\#$11 $\vdash$ x0$q_3$$\#$11 $\vdash$ x0$\#$$q_5$11 $\vdash$ x0$q_6$$\#$x1 $\vdash$ x$q_7$0$\#$x1 $\vdash$ $q_7$x0$\#$x1 $\vdash$ x$q_1$0$\#$x1 $\vdash$ xx$q_2$$\#$x1 $\vdash$ xx$\#$$q_4$x1 $\vdash$ xx$\#$x$q_4$1 $\vdash$ xx$\#$x1$q_R$\\

e) 10$\#$10\\
$q_1$10$\#$10 $\vdash$ x$q_3$0$\#$10 $\vdash$ x0$q_3$$\#$10 $\vdash$ x0$\#$$q_5$10 $\vdash$ x0$q_6$$\#$x0 $\vdash$ x$q_7$0$\#$x0 $\vdash$ $q_7$x0$\#$x0 $\vdash$ x$q_1$0$\#$x0 $\vdash$ xx$q_2$$\#$x0 $\vdash$ xx$\#$$q_4$x0 $\vdash$ xx$\#$x$q_4$0 $\vdash$ xx$\#$$q_6$xx $\vdash$ xx$q_6$$\#$xx $\vdash$ x$q_7$x$\#$xx $\vdash$ xx$q_1$$\#$xx $\vdash$ xx$\#$$q_8$xx $\vdash$ xx$\#$x$q_8$x $\vdash$ xx$\#$xx$q_8$$\sqcup$ $\vdash$ xx$\#$xx$\sqcup$$q_A$




\end{document}
