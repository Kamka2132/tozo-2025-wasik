\documentclass{article}
\usepackage[utf8]{inputenc}
\usepackage{polski}
\usepackage[polish]{babel}
\usepackage{amsmath}
\usepackage{amssymb}

\title{Teoria obliczeń i złożoność obliczeniowa 2025}
\author{Kamila Wasik}
\date{October 2025}

\begin{document}

\maketitle

\section{Zadanie 1}
Dopasowanie to ciąg indeksów $i_1, i_2, \dots, i_k$ taki, że napisy na górze i na dole po złączeniu są identyczne.
Przyjmijmy oznaczenia: $d_1 = [\frac{ab}{abab}], d_2 = [\frac{b}{a}], d_3 = [\frac{aba}{b}], d_4 = [\frac{aa}{a}]$.

Ciąg indeksów: $(1, 2, 3, 1, 2, 4)$

\begin{itemize}
    \item \textbf{Góra:} $ab \cdot b \cdot aba \cdot ab \cdot b \cdot aa = abbabaabbaa$
    \item \textbf{Dół:} $abab \cdot a \cdot b \cdot abab \cdot a \cdot a = abbabaabbaa$
\end{itemize}

Dopasowanie zostało znalezione, ponieważ oba napisy są identyczne.

\section{Zadanie 2}
W alfabecie unarnym każde domino $i$ można opisać parą liczb $(g_i, d_i)$, 
gdzie $g_i$ to liczba jedynek na górze, a $d_i$ na dole.
Dopasowanie istnieje, gdy $\sum g_{i_k} = \sum d_{i_k}$, czyli $\sum (g_{i_k} - d_{i_k}) = 0$.

Algorytm rozstrzygający:\\
1. Jeśli istnieje domino, w którym $g_i = d_i$, zaakceptuj (dopasowanie jednoelementowe).\\
2. Jeśli istnieje domino $i$, gdzie $g_i > d_i$ ORAZ domino $j$, gdzie $g_j < d_j$, zaakceptuj. \\
   Możemy dobrać odpowiednią liczbę domina $i$ i $j$, aby różnice się wyzerowały 
   (rozwiązanie równania $n \cdot (g_i - d_i) + m \cdot (g_j - d_j) = 0$).\\
3. W przeciwnym razie odrzuć.

Skoro algorytm zawsze się kończy, problem jest rozstrzygalny.

\section{Zadanie 3}
\textbf{Dowód (przez redukcję)}\\
Wiemy, że język $ALL_{CFG} = \{ \langle G \rangle \mid L(G) = \Sigma^* \}$ jest nierozstrzygalny.

Redukujemy $ALL_{CFG}$ do $EQ_{CFG}$.\\
Załóżmy, że istnieje maszyna $R$ rozstrzygająca $EQ_{CFG}$. 

Budujemy maszynę $S$ rozstrzygającą $ALL_{CFG}$:\\
$S = $ "Na wejściu $\langle G \rangle$:\\
1. Skonstruuj gramatykę $G_{all}$, która generuje wszystkie słowa nad $\Sigma$ (język regularny $\Sigma^*$).\\
2. Uruchom $R$ na wejściu $\langle G, G_{all} \rangle$.\\
3. Jeśli $R$ akceptuje, zaakceptuj. Jeśli $R$ odrzuca, odrzuć."

Ponieważ $ALL_{CFG}$ jest nierozstrzygalny, $EQ_{CFG}$ również musi być nierozstrzygalny.

\section{Zadanie 4}
Z definicji redukcji $A \le_m B$ istnieje obliczalna funkcja $f$ taka, że:

$w \in A \iff f(w) \in B$.

Z zasad logiki (prawo kontrapozycji i definicja dopełnienia):\\
$w \notin A \iff f(w) \notin B$, 

co jest równoważne:\\
$w \in \overline{A} \iff f(w) \in \overline{B}$.

Ta sama funkcja $f$ świadczy o tym, że $\overline{A} \le_m \overline{B}$.

\section{Zadanie 5}

\textbf{Odpowiedź: Nie}

Funkcja redukująca $f$ jest funkcją \textit{obliczalną}, co oznacza, że może być wykonana przez potężną Maszynę Turinga. 

Niech $A = \{0^n 1^n \mid n \ge 0\}$ (język bezkontekstowy, nieregularny) \\
a $B = \{1\}$ (język regularny).

Możemy zdefiniować funkcję $f(w)$:\\
$f(w) = 1$, jeśli $w \in A$, \\
oraz $f(w) = 0$, jeśli $w \notin A$.

Funkcja ta jest obliczalna, bo Maszyna Turinga może sprawdzić, czy słowo należy do $A$. 

Zatem $A \le_m B$, mimo że $A$ nie jest regularny.

\section{Zadanie 6}
Niech $s_1, s_2, \dots$ będzie listą wszystkich ciągów w $\Sigma^*$. 
Poniższa maszyna Turinga (TM) rozpoznaje język $\overline{E_{\text{TM}}}$.

\begin{quote}
„Na wejściu $\langle M \rangle$, gdzie $M$ jest maszyną Turinga:
\begin{enumerate}
    \item Powtarzaj poniższe kroki dla $i = 1, 2, 3, \dots$
    \item Uruchom $M$ na $i$ kroków dla każdego ciągu wejściowego $s_1, s_2, \dots, s_i$.
    \item Jeśli $M$ zaakceptowała którykolwiek z nich, \textit{zaakceptuj}. W przeciwnym razie kontynuuj.”
\end{enumerate}
\end{quote}


\section{Zadanie 7}
Przypuśćmy nie wprost, że $A_{\text{TM}} \leq_{\text{m}} E_{\text{TM}}$ 
poprzez redukcję $f$. Z definicji redukowalności wyznaczalnej wynika, że 
$\overline{A_{\text{TM}}} \leq_{\text{m}} \overline{E_{\text{TM}}}$ poprzez tę samą 
funkcję redukcji $f$. Jednakże, $\overline{E_{\text{TM}}}$ jest rozpoznawalny 
w sensie Turinga (patrz rozwiązanie zadania 6), a $\overline{A_{\text{TM}}}$ 
nie jest rozpoznawalny w sensie Turinga, co stoi w sprzeczności z Twierdzeniem Rice'a.

\section{Zadanie 8}
Załóżmy, że $A \leq_{\text{m}} B$ oraz $B \leq_{\text{m}} C$. Wówczas 
istnieją funkcje obliczalne $f$ i $g$ takie, że $x \in A \iff f(x) \in B$ oraz 
$y \in B \iff g(y) \in C$. 

Rozważmy złożenie funkcji $h(x) = g(f(x))$. \\
Możemy skonstruować maszynę Turinga, która oblicza $h$ w następujący sposób: \\
- najpierw symuluj maszynę Turinga dla $f$ na wejściu $x$ i nazwij wynik $y$. \\
- następnie symuluj maszynę Turinga dla $g$ na wejściu $y$. 

Wynikiem jest :\\
$h(x) = g(f(x))$. Zatem $h$ jest funkcją obliczalną. Co więcej, 
$x \in A \iff h(x) \in C$. 

Stąd $A \leq_{\text{m}} C$ poprzez funkcję redukcji $h$.

\section{Zadanie 9}
Załóżmy, że $A \leq_{\text{m}} \overline{A}$. \\
Wtedy :\\
$\overline{A} \leq_{\text{m}} A$ poprzez tę samą redukcję wyznaczalną. 

Ponieważ $A$ jest rozpoznawalny w sensie Turinga, Twierdzenie Rice'a implikuje, 
że $\overline{A}$ jest rozpoznawalny w sensie Turinga, a następnie 
Twierdzenie 4.22 implikuje, że $A$ jest rozstrzygalny.

\textbf{TWIERDZENIE 4.22} \\
Język jest rozstrzygalny wtedy i tylko wtedy, gdy jest rozpoznawalny 
w sensie Turinga i jego dopełnienie jest rozpoznawalne w sensie Turinga.

\section{Zadanie 10}
Aby udowodnić równoważność, musimy wykazać dwie implikacje:

1) $(\Rightarrow)$ Załóżmy, że $A$ jest rozpoznawalny przez Maszynę Turinga $M_A$. Definiujemy funkcję redukcji $f(w) = \langle M_A, w \rangle$.
\begin{itemize}
    \item Funkcja $f$ jest obliczalna (wystarczy dopisać stały kod maszyny $M_A$ przed słowem $w$).
    \item Zachodzi warunek: $w \in A \iff \langle M_A, w \rangle \in A_{TM}$. Jeśli $w \in A$, to $M_A$ akceptuje $w$, więc para należy do $A_{TM}$.
\end{itemize}

2) $(\Leftarrow)$ Załóżmy, że $A \le_m A_{TM}$. Wiemy, że język $A_{TM}$ jest rozpoznawalny (przez uniwersalną maszynę Turinga). Ponieważ klasa języków rozpoznawalnych jest zamknięta na redukcje przez odwzorowanie, język $A$ również musi być rozpoznawalny.

\section{Zadanie 11}
($\Leftarrow$) Jeśli $A \le_m 0^*1^*$, to ponieważ $0^*1^*$ jest językiem regularnym (więc rozstrzygalnym), 
a języki rozstrzygalne są zamknięte na redukcje, $A$ musi być rozstrzygalny.

($\Rightarrow$) Jeśli $A$ jest rozstrzygalny, to istnieje maszyna decydująca $M$. 
Definiujemy funkcję redukcji $f(w)$:\\
- Uruchom $M$ na $w$.\\
- Jeśli $M$ zaakceptuje, zwróć słowo $01$ (należy do $0^*1^*$).\\
- Jeśli $M$ odrzuci, zwróć słowo $10$ (nie należy do $0^*1^*$).\\
Funkcja $f$ jest obliczalna i poprawnie redukuje $A$ do $0^*1^*$.

\section{Zadanie 12}
Skoro $B \neq \emptyset$ i $B \neq \Sigma^*$, to istnieje co najmniej jedno słowo $b_{tak} \in B$ oraz co najmniej jedno słowo $b_{nie} \notin B$.

Konstruujemy funkcję $f(w)$:\\
1. Przetestuj, czy $w \in A$ (możemy to zrobić, bo $A$ jest rozstrzygalny).\\
2. Jeśli $w \in A$, zwróć $b_{tak}$.\\
3. Jeśli $w \notin A$, zwróć $b_{nie}$.\\

Ponieważ $w \in A \iff f(w) \in B$, funkcja $f$ jest poprawną redukcją.







\end{document}
